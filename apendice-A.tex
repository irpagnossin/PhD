	Na figura da p�gina \pageref{fig:R-LB} assumimos que os el�tron nos estados de borda desgarrados (em vermelho na figura) percorrem o interior da amostra sem interagir com outros estados de borda at� atingirem o contato $i$, onde as energias se equilibram em $\mu_i$. Supomos isso por simplicidade e porque a presen�a dos plat�s Hall indica condu��o n�o-dissipativa. Ademais, esta configura��o explica nossos resultados, conforme expusemos naquele cap�tulo. Mas curiosamente, aceitar que os el�trons interagem nas regi�es circuladas da figura \ref{fig:R-LB} (nessas regi�es, a probabilidade de sobreposi��o dos estados de borda � maior e, com ela, a de espalhamento) n�o afeta o resultado. O objetivo deste ap�ndice � demonstrar isso e ilustrar ao leitor como escrever e resolver a equa��o de Landauer-Buttiker de forma f�cil.

	A figura \ref{fig:cfg-alternativa} ilustra a nova configura��o, onde introduzimos quatro ``contatos virtuais'' dentro da rede de antipontos (representada pelo ret�ngulo tracejado, por simplicidade). O fluxo de corrente � igual �quele da figura \ref{fig:R-LB}, exceto que agora permitimos haver $\mu_A \ldots \mu_D$.

	Assim, o contato $2$ injeta $N$ estados de borda na amostra e todos chegam no \textit{contato} $A$. O contato $A$ tamb�m injeta $N$ ($ = N + M$) estados, dos quais $M$ atingem $D$ e $N-M$, o contato $B$. E assim por diante. Observe que n�o h� ac�mulo de cargas na amostra, posto que todos os contatos injetam e recebem $N$ estados na amostra, mesmo os virtuais. Com isso em mente, o primeiro passo � definir a matriz de transmiss�o ($T$).

	Fa�a uma tabela com $10$ linhas e $10$ colunas (pois temos dez contatos), conforme abaixo. As colunas indexam os contatos de origem e as linhas, os de destino. Preenchemos a tabela da esquerda para a direita, coluna ap�s coluna. Por exemplo, o contato $1$ ``envia'' $N$ estados para o $2$, de modo que o elemento $T_{21} = N$; o contato $2$ envia $N$ estados para o $A$: $T_{A2} = N$. J� o contato $A$ envia $N-M$ para o $B$ e $M$ para o $D$. Logo, $T_{BA} = N - M$ e $T_{DA} = M$. E assim sucessivamente. Isto nos leva � matriz de transmiss�o.

	\begin{center}
	\begin{tabular}{c|cccccccccc}
		     & 1   & 2   & $A$   & $B$ & 3   & 4   & 5   & $C$   & $D$ & 6   \\
		 \hline
		 1   &     &     &       &     &     &     &     &       &     & $N$ \\
 		 2   & $N$ &     &       &     &     &     &     &       &     &     \\
 		 $A$ &     & $N$ &       &     &     &     &     &       &     &     \\
 		 $B$ &     &     & $N-M$ &     &     &     &     &  $M$  &     &     \\
 		 3   &     &     &       & $N$ &     &     &     &       &     &     \\
 		 4   &     &     &       &     & $N$ &     &     &       &     &     \\
 		 5   &     &     &       &     &     & $N$ &     &       &     &     \\
 		 $C$ &     &     &       &     &     &     & $N$ &       &     &     \\
 		 $D$ &     &     &  $M$  &     &     &     &     & $N-M$ &     &     \\
 		 6   &     &     &       &     &     &     &     &       & $N$ &
	\end{tabular}
	\end{center}

	Agora colocamos o termo $-(N_i + R_i)$ da equa��o \ref{eq:LB} em todos os elementos da diagonal principal. No nosso caso, como nenhum estado � refletido de volta para o contato que o emitiu, $R_i = 0$ sempre. Al�m disso, todos os contatos injetam $N$ estados na amostra, de modo que $N_i = N$, qualquer que seja $i$. Logo, basta colocarmos $-N$ em todos os elementos da diagonal principal.

	Finalmente, escrevemos a tabela em forma matricial, multiplicando-a � direita pela matriz de potenciais dos contatos (elementos $V_i$) e igualando-a � de correntes:

	$$
	\left(\begin{matrix}
	I \\ 0 \\ 0 \\ 0 \\ 0 \\ -I \\ 0 \\ 0 \\ 0 \\ 0 \\
	\end{matrix}\right) 
   =
	\left(\begin{matrix}
		 $-N$ &       &       &     &     &     &     &       &     & $N$ \\
 		 $N$  & $-N$  &       &     &     &     &     &       &     &     \\
 		      &  $N$  & $-N$  &     &     &     &     &       &     &     \\
 		      &       & $N-M$ & $-N$&     &     &     &  $M$  &     &     \\
 		      &       &       & $N$ & $-N$&     &     &       &     &     \\
 		      &       &       &     & $N$ & $-N$&     &       &     &     \\
 		      &       &       &     &     & $N$ & $-N$&       &     &     \\
 		      &       &       &     &     &     & $N$ & $-N$  &     &     \\
 		      &       &  $M$  &     &     &     &     & $N-M$ & $-N$&     \\
 		      &       &       &     &     &     &     &       & $N$ & $-N$
	\end{matrix}\right) \cdot
	\left(\begin{matrix}
	V_1 \\ V_2 \\ V_A \\ V_B \\ V_3 \\ V_4 \\ V_5 \\ V_C \\ V_D \\ V_6 \\
	\end{matrix}\right)
	$$

, sendo que $I_1 = -I_4 = I$ e os outros elementos s�o nulos
