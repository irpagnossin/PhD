
\section{Magnetoresist�ncia}
	*definir frequencia ciclotronica na lista de simbolos
	Vers�o topol�gica do efeito Hall: Chern number

	Um el�tron sujeito a um campo magn�tico $H$ percorre uma trajet�ria circular de raio $v/\omega_c$ (raio de Larmor), onde $\omega_c$ � a \emph{freq��ncia ciclotr�nica}\footnote{Ou seja, a velocidade angular com que o el�tron percorre essa trajet�ria.} e $v$ � a velocidade do el�tron. Quando $H$ � suficientemente intenso, a circunfer�ncia da trajet�ria torna-se compar�vel � extens�o da fun��o de onda do el�tron (limite qu�ntico), de tal forma que ela s� pode assumir valores m�ltiplos inteiros do comprimento de onda de de Broglie\rlap{.}\footnote{O argumento aqui � id�ntido �quele proposto por Bohr para explicar a quantiza��o das �rbitas eletr�nicas nos �tomos.} Por conseguinte, a energia cin�tica torna-se quantizada

	\textbf{O modelo de Landau.} 

	A densidade de estados do sistema bidimensional de el�trons sob a influ�ncia de um campo magn�tico (perpendicular ao plano do \textsc{2des})

\textbf{Preciosismo:} � comum na literatura cient�fica falar-se no ``espectro \textsc{fft}'' de $R_{xx}(H)$, mas deve-se lembrar que a \textsc{fft} � um \textbf{algor�tmo} de transformada de Fourier.\cite{Ridley:1993}

	Note que, conforme~\cite{Roth:1966}, as oscila��es da resistividade tamb�m s�o observadas em $\rho_{xy}$. Veja a minha tese de mestrado.
