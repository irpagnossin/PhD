

	Este trabalho n�o teria sido poss�vel sem o apoio da minha esposa, \textbf{Fabiana Aparecida Cagnotto de Farias}, do meu filho, \textbf{Giovani Cagnotto Pagnossin}, e dos meus pais, \textbf{Marco Ant�nio Pagnossin} e \textbf{Luci Andrade Ramos Pagnossin}. Por este motivo agrade�o-os todos e compartilho com eles, em cotas iguais, os m�ritos deste trabalho.

	Agrade�o � minha eterna orientadora e amiga, a prof\rlap{.}� \textbf{Euzi Concei��o Fernandes da Silva}, por acreditar em mim (mesmo quando eu n�o merecia) e orientar-me sempre sobre tudo e mesmo de longe.

	Ao meu orientador, prof. \textbf{Guannadii Michailovich Gusev}, por aceitar-me como seu aluno quando a prof\rlap{.}� Euzi foi para os Estados Unidos e por me indicar ``o caminho das pedras\rlap{.}''

	Aos amigos \textbf{Alain Andr� Quivy}, \textbf{Angela Mar�a Ortiz de Zevallos M�rquez}, \textbf{Celso de Ara�jo Duarte}, \textbf{Francisco de Paula Oliveira}, \textbf{Jos� Geraldo Chagas}, \textbf{Luis Enrique Gomez Armas}, \textbf{M�rcia Ribeiro}, \textbf{M�nica Jimenez Clauzet}, \textbf{Niko Churata Mamani}, \textbf{Tatiana Lacerda} e \textbf{Tomas Erikson Lamas}, agrade�o pela preciosa amizade e pelos incont�veis aux�lios.

	� \textbf{Rita} (EPUSP), agrade�o por disp�r-se sempre a executar a grava��o da barra-Hall, em todas as amostras.

	Ao professor \textbf{Ant�nio Carlos Seabra} (EPUSP) sou grato pelo acesso ao microsc�pio eletr�nico utilizado neste trabalho; e �s colegas \textbf{Mariana Pojar} e \textbf{Simone Camargo Trippe}, por ensinarem-me a utiliz�-lo.

	Ao professor \textbf{Iouri Poussep} (IFSC), pelo acesso ao seu laborat�rio; e ao seu aluno \textbf{Haroldo Arakaki}, por auxiliar-nos nos processos de litografia e evapora��o de ouro. Processos similares foram feitos tamb�m no Instituto de F�sica da UNICAMP, pelos qual sou grato ao \textbf{Antonio Augusto de Godoy Von Zuben} (\textbf{Tot�}).

	\foreign{I would like to thank prof. \textbf{Ajit Kumar Meikap}, who taugh me the principles of weak localization and with who I had trainning my spoken english skill before my trip to France; also for the colaboration on many studies.}

	\foreign{Je remercie au prof. \textbf{Jean-Claude Portal}, pour m'admettre dans son �quipe de travail en France, o� j'ai �tabli contact avec d'autres paradigmes. De la m�me fa�on, je remercie aux coll�gues \textbf{Vincent Renard} et \textbf{Sami Sassini}, qui m'ont accueilli dans le pays et m'ont enseign� a conduire les �quipements, sans rien demander en exchange.}

	Finalmente, sou grato � \textbf{FAPESP}, por ter disponibilizado importantes recursos p�blicos para este projeto, cujos resultados n�o podem ser representados em sua totalidade nessas poucas p�ginas, por mais que eu tente.
