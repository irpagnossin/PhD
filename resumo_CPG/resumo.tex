\documentclass{article}
	\usepackage[latin1]{inputenc}
\begin{document}

	\Large\textbf{Resumo}
	\normalsize

		Apresentamos neste trabalho algumas propriedades do transporte de cargas de heteroestruturas contendo pontos-qu�nticos. Tr�s t�picos foram explorados: no primeiro, observamos um comportamento an�malo nos plat�s do efeito Hall qu�ntico, que atribu�mos � exist�ncia de estados de borda contra-rotativos; no segundo, determinamos o tempo de decoer�ncia do sistema bidimensional de el�trons em fun��o do est�gio evolutivo de pontos-qu�nticos de InAs autoformados nas suas proximidades. Conclu�mos que a tens�o mec�nica acumulada durante o crescimento epitaxial ``congela'' os el�trons, reduzindo a taxa de decoer�ncia; finalmente, testamos algumas das poss�veis configura��es de heteroestruturas visando a constru��o de fotodetectores baseados em pontos-qu�nticos. Observamos que a repeti��o da regi�o-ativa pode ser utilizada como um par�metro no controle das mobilidades qu�nticas e, por conseguinte, das propriedades de opera��o desses detectores.

	In this work we present transport properties of heterostructures with quantum-dots. Three subjects were exploited: on the first one, we observed anomalous quantum Hall plateaus, for wich we attributed to the existence of counter-rotating edge-states; on the second subject, we determined the decoherence time of the bidimensional electron system as a function of the evolutionary stage of nearby self-assembled quantum-dots. We concluded the mechanical stress accumulated during the epitaxial growth "freezes" the electrons, reducing the decoherence rate; finally, we tested some base-heterostructures of infrared photodetectors. We observed the stacking of active-regions can be used as a parameter to control quantum-mobilities and, as a consequence, the operation properties of such detectors.

\end{document}
